\documentclass[10pt,authoryear,onecolumn]{article}
\usepackage[utf8]{inputenc}


\usepackage[pagebackref=true,colorlinks=true]{hyperref}	% Links
\usepackage{amsmath}
\usepackage{siunitx}	% Valores con unidades
	\sisetup{range-phrase = --}
	\sisetup{separate-uncertainty = true}
	\sisetup{allow-number-unit-breaks = true}
	\sisetup{tight-spacing = true}
\usepackage{soulutf8}		% Resaltado, subrayado y demás
	\DeclareRobustCommand{\hlcolor}[2][yellow]{{\sethlcolor{#1}\hl{#2}}}

\usepackage{amssymb}
\DeclareSIUnit\ppm{ppm}
\DeclareSIUnit\pixel{px}
\DeclareSIUnit\arb{AU}

\newcommand{\I}{\mathrm{i}}
\newcommand{\FE}{\textit{FE}}
\newcommand{\Sum}{\textit{Sum}}
\newcommand{\R}{\textit{R}}
\newcommand{\DefX}{\textit{DefX}}
\newcommand{\DefY}{\textit{DefY}}
\newcommand{\low}[1]{\textsubscript{#1}} % Subíndices en modo texto


% No importar las sigiuentes librerías en caso de usar la plantilla de Elsevier
\usepackage{authblk}
\usepackage{natbib,hyperref}
\usepackage{graphicx}


\begin{document}


%% Formato Elsevier


%\journal{Journal of Petroleum Science and Engineering}
%
%\begin{frontmatter}
%
%\title{Online oil-in-water and suspended solids measurement device based on thermal lens spectrometry and forward scattering}
%
%\author[AddressDF]{Axel Lacapmesure\corref{Corr1}} \ead{axel.lacapmesure@gmail.com}
%\author[AddressYTEC]{Darío Kunik}
%\author[AddressFIUBA]{Oscar E. Martínez}
%
%\address[AddressDF]{Departamento de Física, Facultad de Ciencias Exactas y Naturales, Universidad de Buenos Aires, Ciudad Autónoma de Buenos Aires, Argentina}
%\address[AddressYTEC]{YPF Tecnología S.A., Buenos Aires, Argentina}
%\address[AddressFIUBA]{Laboratorio de Fotónica, Facultad de Ingeniería, Universidad de Buenos Aires, Ciudad Autónoma de Buenos Aires, Argentina}
%
%\cortext[Corr1]{Corresponding author}
%
%
%
%\begin{abstract}
%
%\end{abstract}
%
%
%\begin{keyword}
%
%\end{keyword}
%
%\end{frontmatter}

%% Formato típico

\title{Online oil-in-water and suspended solids monitor based on thermal lens spectrometry and forward scattering}
\author{}
\date{}

\maketitle

\section*{Highlights}

\begin{itemize}
	\item Fast measurements with no sample preparation needed.
	\item \SI{0.02}{\ppm} resolution in oil concentration measurements.
	\item Suspended solids and oil droplets are detected separately through signal processing.
	\item Particles of different sizes are distinguished by peak height distribution.
	\item Particles concentrations are obtained from interarrival times.
	\item The device proves ideal as an online monitor incorporated into injection flowline.
\end{itemize}

\end{document}

\endinput